\documentclass[11pt, a4paper, leqno]{article}
\usepackage{a4wide}
\usepackage[T1]{fontenc}
\usepackage[utf8]{inputenc}
\usepackage{float, afterpage, rotating, graphicx}
\usepackage{epstopdf}
\usepackage{longtable, booktabs, tabularx}
\usepackage{fancyvrb, moreverb, relsize}
\usepackage{eurosym, calc}
% \usepackage{chngcntr}
\usepackage{amsmath, amssymb, amsfonts, amsthm, bm}
\usepackage{caption}
\usepackage{mdwlist}
\usepackage{xfrac}
\usepackage{setspace}
\usepackage[dvipsnames]{xcolor}
\usepackage{subcaption}
\usepackage{minibox}
% \usepackage{pdf14} % Enable for Manuscriptcentral -- can't handle pdf 1.5
% \usepackage{endfloat} % Enable to move tables / figures to the end. Useful for some
% submissions.
\usepackage[
    natbib=true,
    bibencoding=inputenc,
    bibstyle=authoryear-ibid,
    citestyle=authoryear-comp,
    maxcitenames=3,
    maxbibnames=10,
    useprefix=false,
    sortcites=true,
    backend=biber
]{biblatex}
\AtBeginDocument{\toggletrue{blx@useprefix}}
\AtBeginBibliography{\togglefalse{blx@useprefix}}
\setlength{\bibitemsep}{1.5ex}
%\addbibresource{../../paper/refs.bib}

\usepackage[unicode=true]{hyperref}
\hypersetup{
    colorlinks=true,
    linkcolor=black,
    anchorcolor=black,
    citecolor=NavyBlue,
    filecolor=black,
    menucolor=black,
    runcolor=black,
    urlcolor=NavyBlue
}


\widowpenalty=10000
\clubpenalty=10000

\setlength{\parskip}{1ex}
\setlength{\parindent}{0ex}
\setstretch{1.5}


\begin{document}

\title{Macroeconomic consequences of eliminating social security in the US: an OLG model analysis}

\author{Lorenzo Zecca\thanks{Bonn University. Email: \href{mailto:s78lzecc@uni-bonn.de}{\nolinkurl{s78lzecc [at] uni-bonn [dot] de}}.}}

\date{
    31st March 2023
}

\maketitle


\begin{abstract}
    This project replicates the study of \hyperlink{Conesa and Kruger 1999}{Conesa and Kruger (\textcolor{blue}{1999})}
    : we consider a discrete time overlapping generations model, 
    where the economy is populated by a continuum with given mass 
    growing at a constant rate $n$ of ex-ante identical individuals.
    We compare two steady states: 
    one in which the government runs a social security system, financed 
    through taxes on labor; and another one, where the there is no public 
    pension system, and eranings from labor are not taxed. However, we will cover only 
    the comparion between the two steady states, neglecting the transition dynamics analysis.
\end{abstract}

\clearpage

\section{Social Security Reform}
\hyperlink{Conesa and Kruger 1999}{Conesa and Kruger (\textcolor{blue}{1999})} ask: What are the macroeconomic and welfare
consequences of eliminating a pay-as-you-go social security system in the
U.S.?
The exact experiment is as follows. The economy is in a initial (i.e., pre-reform) 
steady-state equilibrium with the social security tax rate equal to
$11\%$. All of a sudden, the government decides to eliminate the social security
program. In the context of our model, the social security tax rate, $\tau$ , changes
exogenously from $11\%$ to $0\%$. Through the government budget constraint, this policy
 change implies that the pension benefit b becomes 0. We
then analyze the final (i.e., post-reform) steady-state equilibrium in which
the social security tax rate is $0\%$, while all other parameters shown in Table 1 remain unchanged. Importantly, we ignore the transitional dynamics from
the initial steady state to the final steady state (you can find an extensive
discussion of transitional dynamics results in the paper)


\section{Numerical results} % (fold)
The figures below show life-cycle profiles of labor supply, assets, 
consumption, and earnings in the two steady states of
our model. However, the images are obtained setting only 2 iterations.
 A proper analisys would indeed need around 30 to get reasonable results.
This project uses only 2 to speed up the code and show that the project itself is running.

\begin{table}[!htb]
    \begin{center}
    \begin{tabular}{lccc}
      \textbf{Variable}  & \textbf{Initial Steady State} & \textbf{Final Steady State} & \textbf{Change} \\
    \midrule
    \textit{Pension benefit and prices:}                          &         &           &        \\
    \textbf{$b$}                          &   0.2      &    0       &    -    \\
    \textbf{$w$}                          &    1.43     &     1.54      &   7.69\%     \\
    \textbf{$r$}                          &    2.65\%     &    1.54\%       &    -1.01 pt. pt.    \\
    \textbf{($1-\tau)w$}                          &   1.27      &   1.54        &  21.26\%      \\
    \midrule
    \textit{Efficiency Measures:}                          &         &           &        \\
    \textbf{$L$}                          &   0.303      &     0.32      &   5.61\%     \\
    \textbf{$K$}                          &   2.818      &      3.677     &   30.48\%     \\
    \textbf{$K/L$}                          &     9.3    &     11.49      &   23.55\%     \\
    \textbf{$Y$}                          &    0.67     &     0.77      &   15\%     \\
    \midrule
    \textit{Welfare:}                          &         &           &        \\
    \textbf{$V_{1}(0)$}                          &    -52.52     &     -49.82      &  -      \\
    \textbf{CEV}                          &  -       &     12.82\%      &   -     \\
    
    \bottomrule
    \end{tabular}
    \caption{Aggregate Variables, Prices and Welfare Across
    Steady States. Results obtained using the dataset from 2014 to 2018, and setting 50 iterations.
    Notes: Initial steady state with social security $(\tau = 11\%)$, Final steady state without
    social security $(\tau = 0\%)$, $b$ – pension benefit, $r$ – interest rate, $w$ – wage, $L$ – aggregate effective labor supply, $K$ - aggregate saving, $Y$ – aggregate output,
    $ K/L$ – capital intensity, $V_{1}(0)$ – indirect utility of a newborn
    agent with zero assets, CEV – consumption equivalent variation measure for a newborn
    household with zero assets}
    \end{center}
\end{table}

\clearpage

\section{Aggregate variables' profile: Initial vs. Final Steady State}
\begin{figure}[ht]

\centering
\includegraphics[width=0.85\textwidth]{../bld/python/figures/steady_states/savings_by_age.png}
\caption{\emph{Python:} Savings by age. Larger in the final ss: without social security, agents have to accumulate more wealth.}
\label{fig:python-K}
\end{figure}

\begin{figure}[ht]
    
    \centering
    \includegraphics[width=0.85\textwidth]{../bld/python/figures/steady_states/labor_by_age.png}
    \caption{\emph{Python:} Effective labor supply by age.}
    \label{fig:python-L}

\end{figure}

\begin{figure}[ht]

    \centering
    \includegraphics[width=0.85\textwidth]{../bld/python/figures/steady_states/earnings_by_age.png}
    \caption{\emph{Python:} Earnings by age. Larger in the final ss: agents work more and at higher wages.}
    \label{fig:python-E}

\end{figure}


\begin{figure}[ht]

    \centering
    \includegraphics[width=0.85\textwidth]{../bld/python/figures/steady_states/consumption_by_age.png}
    \caption{\emph{Python:} Consumption by age.}
    \label{fig:python-C}

\end{figure}

\clearpage

\section{Conclusions}
In conclusion,... 

\clearpage

\begin{thebibliography}{}

\bibitem{} \hypertarget{Conesa and Kruger 1999}{Conesa, J. and Krueger, D. (1999).} "Social 
    Security Reform with Heterogeneous Agents." \textit{Review of Economic Dynamics, 2, p. 757-95}\
\end{thebibliography}

\end{document}